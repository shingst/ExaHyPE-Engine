\documentclass{article}
\usepackage{amsmath,hyperref}

\begin{document}

\title {Testing basic ExaHyPE scaling} 

\maketitle

\begin{abstract}
Mini benchmark to understand when scaling fails. With a synthetic
PDE. Developed by Anne, Luke, Sven.
\end{abstract}


\section{The most basic PDE}

We decide for the \emph{fully nonconservative} PDE
\begin{equation}\label{pde}
\partial_t Q + B(Q)\nabla Q = S(Q)
\end{equation}
%
with the Nonconservative product NCP $B\nabla Q$ and the
algebraic source term~$S$. We don't want to implement any
sensible physics equation but instead solve a kind of
\emph{dummy} problems when computing the PDE contributions.

Therefore, the physical content of the initial data is not
relevant. $\vec Q=0$ is a good choice. For the boundary
conditions, the same is true, zero boundary conditions,
outflow or exact boundary conditions are applicable.

\section{Parameter space}

The parameter space spawned by these five independent
quantities shall be explored:

\begin{align}
n& \in \{ 1,5,7,9,20, 60 \} \quad\text{unknowns}
\\
C_S &\in \{ 1, 10^2, 10^3, 10^4 \} \quad\text{FLOPS}
\\
C_{NCP} &\in \{ 1, 10^2, 10^3, 10^4 \} \quad\text{FLOPS}
\\
\text{grid} &\in  \{ \text{unigrid}, \text{spherical limiting} \}
\\
\text{scaling} &\in  \{ \text{Single core}, \text{Single node TBB}, \text{Hybrid TBB/MPI} \}
\end{align}
%
\subsection{Meaning of parameters}
\begin{itemize}
\item $n$ unknowns means that $\dim(Q)=n$. It is a direct measure
for the memory requirements of the simulation and serves as a
stress-test for the ADER scheme and MPI communication.

\item $C_S$ and $C_{NCP}$ indicates the costs of the synthetic
implementation of the $S(Q)$ and $NCP(Q,\nabla Q)$ functions,
respectively.

\item Unigrid means uniform grid. spherical limiting means the
physical admissiblity criterion at
\url{http://dev.exahype.eu/astrophysics-userguide/html/Benchmarks/ns.html#limiter-geometry},
a simple idealized limiting criterion.

\item Scaling means \emph{strong scaling}. First question to answer
is whether the simulation runs at all, then to measure the single
core performance (physical time over simulation time per degree
of freedom) and then to compare this quantity to exploiting a
full node (say Iboga, 20 cores). Hybrid scaling by means of one
rank per node. We run up to nodes
$\in \{ 1+3 = 4, 1+3^2 = 10, 1+3^3 = 28 \}$.
\end{itemize}

\subsection{Fixed parameters}
\begin{itemize}
\item  We use a base grid of $3^3$ cells per axis, giving us in total 19683 cells (at unigrid).
\item  LimitingADERSolver with 2nd order FV solver (MUSCL) and DMP observables = 0.
\item  Always Dim3.
\item  Always polynomial order $n=3$
\item  No plotting at all.
\item  Always compile with TBB and MPI enabled.
\item  Always compile with NCP and algebraicSource enabled
(implement empty functions when $C_S=0$ or $C_{NCP}=0$). That is,
the scheme shall always implement eq \eqref{pde}.
\item  Run up to $T = 10 ~\text{iteration}$.
\end{itemize}

\section{Implementation}
We chose to implement an Ackermann function to artificially consume
FLOPs, and it is also a function that the compiler shouldn't be able to
optimise away. A table of values is here 
\begin{center}
\begin{tabular}{c|c|c}
input & itr  & $\sim$ itrs\\ 
\hline
F(1,1)& 3 & $1$\\
F(2,2)& 54 & $10$\\
F(2,3)& 110 & $10^{2}$\\
F(3,2)& 3903 & $10^{3}$\\
F(3,3) & 34386 & $10^{4}$
\end{tabular}
\end{center}

\section{Results}
The results in this section were run on iboga which has 20 cores per
node. This means that for 28 MPI ranks, there has to be internode
communication. Perhaps testing this on supermuc would be interesting.

\subsection{Unigrid}

The domain of the grid size is $1000\times1000\times1000$ with 
\texttt{maximum-mesh-size} $=40$.

\begin{center}
\begin{tabular}{c|c|c|c}
params & $n=60$ & $C_{S} = 1$ & $C_{NCP} = 1$\\
\hline
ranks & 1 & 20 & 28\\
\hline
walltime (s) & 98.99 & 49.37 & 12.01
\end{tabular}
\end{center}

\begin{center}
\begin{tabular}{c|c|c|c}
params & $n=60$ & $C_{S} = 10^{2}$ & $C_{NCP} = 10^{2}$\\
\hline
ranks & 1 & 20 & 28\\
\hline
walltime (s) & 388.17 & 193.25 & 47.67
\end{tabular}
\end{center}

\begin{center}
\begin{tabular}{c|c|c|c}
params & $n=60$ & $C_{S} = 10^{3}$ & $C_{NCP} = 10^{3}$\\
\hline
ranks & 1 & 20 & 28\\
\hline
walltime (s) & - & 1480.41 & 371.37
\end{tabular}
\end{center}

For these cases, there is always a speedup of roughly 8.


\end{document}
